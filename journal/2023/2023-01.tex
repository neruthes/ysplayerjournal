\documentclass[11pt,a4paper]{report}
\usepackage[a4paper,textwidth=42em,tmargin=15mm,bmargin=28mm,footskip=15mm]{geometry}

\usepackage{calc}
\usepackage[dvipsnames]{xcolor}
\usepackage{amsmath,amssymb,xltxtra,fontspec,xunicode,tocloft,fancyhdr,indentfirst}
\usepackage{graphicx,eso-pic,datetime2,tabu,tcolorbox,multicol}
\graphicspath{ {./large-assets}{./large-assets/journal/2023} }
\usepackage{paralist,enumitem,varwidth}
\setdefaultleftmargin{5em}{2em}{1em}{1em}{1em}{1em}

\usepackage[hidelinks]{hyperref}
\hypersetup{
	pdfauthor={Neruthes 等各个作者},
	colorlinks=false,
	pdfpagemode=FullScreen
}

\usepackage[PunctStyle=plain,RubberPunctSkip=false,CJKglue=\hskip 0pt,CJKecglue=\hskip 2.5pt plus 20pt]{xeCJK}
\usepackage{xeCJKfntef}
% \newcommand{\CJKecglue}{\hskip 2.5pt plus 20pt}

\setmainfont{BaskervilleF}
\setromanfont{BaskervilleF}
\setsansfont{Mona-Sans}
\setmonofont{JetBrains Mono NL}
\setCJKmainfont{Noto Serif CJK SC}
\setCJKromanfont{Noto Serif CJK SC}
\setCJKsansfont{Noto Sans CJK SC}
\setCJKmonofont{Noto Sans CJK SC}
% =========================================


\linespread{1.2}
\setlength{\parindent}{2em}
\setlength{\parskip}{0pt}
\setlength{\columnsep}{2em}
% \setlength{\columnseprule}{0.4pt}

% Playing with tocloft
\renewcommand{\contentsname}{\LARGE\sffamily\bfseries 本期看点~~~~\Large\mdseries TBALE OF CONTENTS}
\renewcommand{\cftchapfont}{\sffamily\small}
\settowidth{\cftchapnumwidth}{\cftchapfont 2222}
\renewcommand{\cftchappagefont}{\sffamily\small}
\setlength{\cftbeforechapskip}{1pt}









\newcommand{\isDraft}[0]{
	\AddToShipoutPictureFG*{%
		\put(30mm,30mm){%
			\rotatebox{45}{\textcolor{red}{\fontsize{100pt}{100pt}\selectfont\rmfamily D~R~A~F~T}}
		}
	}
}
\newcommand{\maketitlepage}[7]{
	% argv: year, seq, textcolor, bgpicshift, issuetime, bgimg, briefing
	\hypersetup{pdftitle={尘世七国的记忆 - #5}}
\begin{titlepage}
	\newgeometry{textwidth=\textwidth+2em,tmargin=16mm,bmargin=16mm}
	\parindent=0pt
	\AddToShipoutPictureBG*{%
		\put(#4,-1mm){%
			% \includegraphics[height=\paperheight+2mm]{cover/cover-#1-#2}
			\includegraphics[height=\paperheight+2mm]{#6}
		}
	}
	\color{#3}
	\rule{\linewidth}{2.5pt}\vskip 15pt
	\begin{minipage}{\linewidth}
		\fontsize{45pt}{50pt}\selectfont
		\rmfamily\bfseries
		尘\hfill 世\hfill 七\hfill 国\hfill 的\hfill 记\hfill 忆
	\end{minipage}\par\vskip 10pt
	\begin{minipage}{\linewidth}
		\normalsize
		\sffamily\bfseries
		那些关于回旋镖、照妖镜、时光机的故事\hfill
		\LARGE#5
	\end{minipage}\par\vskip 10pt
	\rule{\linewidth}{2.5pt}\vskip 55pt

	\begin{minipage}{\linewidth}
		#7
	\end{minipage}

	\vfill
	\begin{minipage}{\linewidth}
		\footnotesize\sffamily
		主编 Neruthes\hfill 文章版权归作者\hfill CC BY-NC-ND 4.0
	\end{minipage}
	\rule{\linewidth}{2.5pt}
\end{titlepage}
\restoregeometry
\color{black}
\fancypagestyle{plain}{
    
	\renewcommand{\headrulewidth}{0pt}
	\renewcommand{\footrulewidth}{0.4pt}
	\lhead{}
	\chead{}
	\rhead{}
	\lfoot{\footnotesize \textbf{尘世七国的记忆}\hskip 2em#5}
	\cfoot{}
	\rfoot{\small\thepage}
}
\pagestyle{plain}\sffamily
}





\newcommand{\coverstory}[2]{
	% argv: author, title
	\begin{center}
		\sffamily\small\bfseries 封面故事~~/~~#1\\
		\rmfamily\Huge#2
	\end{center}
}
\newcommand{\dividearticles}[0]{
	\vskip 30pt plus 30pt minus 5pt
	\noindent\rule{\linewidth}{3.5pt}
	\vskip 25pt plus 25pt minus 5pt
}
\newcommand{\iarticle}[4]{
	% argv: author, title, tagline, noteline
	% \vskip 50pt
	\stepcounter{chapter}
	\setcounter{section}{0}
	\phantomsection\addcontentsline{toc}{chapter}{\numberline{\thechapter}#2}
	\noindent\parbox{\linewidth}{
		{\small\sffamily\bfseries#3}\par\vskip 7pt\nopagebreak
		{\LARGE\rmfamily\bfseries#2}\par\vskip 3pt\nopagebreak
		{\footnotesize\rmfamily\mdseries#4}\par\nopagebreak
		\rule{\linewidth}{0.4pt}\nopagebreak
	}\par\nopagebreak
	\vskip 15pt\nopagebreak
}
\newcommand{\isecnum}[0]{
	\stepcounter{section}
	\mbox{\hbox to 21.9pt{\arabic{section}. \hfill}}
}
\newcommand{\isection}[1]{
	% argv: sectionname
	\vskip 22pt plus 40pt minus 5pt
	\noindent\parbox{\linewidth}{\Large\rmfamily\bfseries#1}
	\vskip 15pt plus 30pt minus 5pt
}
% \newcommand{\notebox}[2]{
% 	\begin{tcolorbox}[colback=gray!5!white,colframe=gray,boxrule=0.4pt,width=\linewidth,left=5pt,right=5pt,top=5pt,bottom=5pt,sharp corners]
% 		\fontsize{10pt}{11pt}\selectfont
% 		#2
% 	\end{tcolorbox}\par
% }
\newcommand{\paraimg}[1]{%
        \vskip 0pt plus 30pt
        \noindent\includegraphics[width=\linewidth]{#1}
        \vskip 3pt plus 30pt
    }
\newcommand{\flisthref}[2]{%
    \noindent%
    \parbox{\linewidth}{
        \raggedright
		\fontsize{9pt}{12pt}\selectfont%
		\baselineskip=10pt
		\href{#1}{\rmfamily#2}\hfill\nopagebreak%
		\\
		\fontsize{8pt}{8.5pt}\selectfont%
		\href{#1}{\sffamily\selectfont\mbox{\textcolor{blue!35!gray}{[~#1~]}}}%
		\hfill
	}\vskip 5pt
}




\newcommand{\stdlastpage}[2]{
	% argv: yearRange, extraContent
	\clearpage\pagestyle{empty}
	\leavevmode\vfill
	\noindent\begin{minipage}{382.5pt}
		\fontsize{8.5pt}{10.5pt}\selectfont
		\parindent=0pt
		Copyright \copyright{} #1 Neruthes and other contributors.\\
		版权所有 \copyright{} #1 Neruthes 与其他贡献者。\\
		文章版权归各个作者所有。作者文章内容不代表本刊立场。\\
		本刊独立运作,与米哈游没有任何隶属、指导、赞助关系。\\
		此文档内包含《原神》游戏画面之截图、官方宣传图片,其版权属于上海米哈游影铁科技有限公司或其关联单位。
		本刊在版权法「合理使用」框架内谨慎使用此类素材。
		此文档内可能包含其他图片,具体版权情况以此文档内的备注为准。\\
		此文档以\CJKecglue\href{https://creativecommons.org/licenses/by-nc-nd/4.0/}{\CJKunderline{CC BY-NC-ND 4.0}}\CJKecglue{}许可证发布。\\
		任何人和单位可以在非商业使用、不修改的前提下复制、印刷、分发此文档(PDF)。\par
		\vskip 10pt

		本期刊在GitHub上的项目仓库:\href{https://github.com/neruthes/ysplayerjournal}{https://github.com/neruthes/ysplayerjournal}\\
		有关投稿方式等信息,请参阅项目仓库首页的README.md文件。
		\par
		\vskip 10pt

		#2
	\end{minipage}
}
\newcommand{\slashsep}[0]{\hskip 7pt/\hskip 7pt}








% ============================
% Miscellaneous utilities
\newcommand{\cjkul}[1]{\CJKunderanyline{0.5ex}{\rule{2pt}{0.4pt}}{#1}}
\newcommand{\pozhehao}[0]{{\fontspec{FandolSong}\CJKfontspec{FandolSong}——}}
% \newcommand{\pozhehao}[0]{\mbox{\raisebox{0.5ex}{\rule{2em}{0.4pt}}}}


\begin{document}
\isDraft
\maketitlepage{2023}{01}{white}{-195mm}{2023 年 01 月~~(上)}{
	\coverstory{Neruthes}{主编创刊寄语}
}
\tableofcontents\clearpage






\iarticle{Neruthes}{主编创刊寄语}{}{作者~/~Neruthes \hfill 2022/12/17}

% \begin{multicols*}{2}
\section*{记录的意义}
2022年就要结束了,提瓦特的旅途大约已经过半,我们在蒙德、璃月、稻妻、须弥留下了弥足珍贵的回忆。
在游戏本体之外,节奏和4D剧情是游戏乐趣的重要组成部分,同样值得铭记。
在魔神任务第3章第5幕、魔神任务间章第3幕,米哈游向我们展现了岁月史书巨大威力。
将历史真相(包括争议中不同观点对撞的场面)记录下来,不仅有吃瓜的美味,更是遨游星海之人与摩拉克斯的契约。

\section*{个人的爱好}
今年,我更加深入地体验了LaTeX,这是比HTML+CSS更加优雅、更加强大的排版工具链。
虽然之前\href{https://neruthesgithubdistweb.vercel.app/qyxt/qyxt/qyxt-2022-02.pdf}{\CJKunderline{关于促进群友交流的尝试}}%
一直未能达到预期的效果,但那段经历让我对LaTeX工具链的熟悉程度提升了很多。

\section*{欢迎来投稿}

本刊接受如下投稿方式:

\begin{compactenum}
	\item \href{https://github.com/neruthes/ysplayerjournal}{\CJKunderline{在GitHub开issue}},将投稿内容写入正文。
	\item 在Telegram上联系\CJKecglue\href{https://t.me/neruthes}{\CJKunderline{@neruthes}},提供稿件内容(请不要发附件)。
	\item 以纯文本形式向\CJKunderline{i@neruthes.xyz}发送电子邮件,将投稿内容写入正文。
\end{compactenum}

向本刊投稿,即代表你接受以下条件:

\begin{compactenum}
	\item 你同意你对本刊作出的贡献将会以上文「版权声明」章节规定的形式向第三方授权。
	\item 本刊在接受你的投稿时可能对文本内容做出小幅修订(包括但不限于:错别字,拼写错误,成语误用)。
	\item 本刊主编有权指定下一任主编,由其继承主编在维护、编撰本刊范围内形成的一切权利(包括但不限于:指定继任者的权利,基于你过去的投稿内容推出新期刊、修订往期期刊、编撰合订本的权利)。
\end{compactenum}
% \end{multicols*}







\iarticle{Neruthes}{Bilibili 用户「我牌哪去了」年底视频瓜}{}{作者~/~Neruthes \hfill 2022/12/17}


\begin{multicols*}{2}
	[\noindent 我们将事件的主角简称为「小床」。]

	\section*{前情提要:开播原神之前}
	小床在开播原神之前有另一个账号「我床哪去了」(所以由此得到「小床」之昵称)。老账号在开播原神前已经积累大约7.5万关注者。
    老账号在2022年9月3日发表视频\href{https://www.bilibili.com/video/BV12e4y1Y71p}{\cjkul{《这下不得不聊一下原神了...》}}\hfill\footnote{BV12e4y1Y71p}\hfill。
    该视频内容比较「理中客」,站稳「局外人」立场,不评游戏只评节奏,看起来像是打算试探其既有粉丝是否支持向原转型。然后告一段落。

    小床建号「我牌哪去了」,并于2022年11月30日发表了第一个视频%
    \href{https://www.bilibili.com/video/BV1XG411M7Nh}{\cjkul{《原神【七圣召唤】不开天梯是好事吗?》}}\hfill\footnote{BV1XG411M7Nh}\hfill。
    这个视频延续之前的风格,属于「局外人臆测」的立场,客观地或至少看似客观地探讨了天梯等机制在七圣召唤中的取舍问题。
    
    此视频内容可称无功无过,但与当时各个玩家社区的天梯话题之争议漩涡形成了共鸣,所以引发了一点争议。当时天梯、段位话题下主要有两派意见:
    一派认为,游戏内排位属于资格证,没有段位会让玩家社区难以基于发言人的段位评估其战术理解的参考价值,让社区的交流更容易被低段位玩家「污染」,
    并且不喜欢PvE的传统米氏PvE玩家可以选择不玩天梯排位只打快速匹配;
    另一派认为,米哈游不忘初心,始终记得原神是一个PvE游戏。

    \section*{第一阶段:为七圣召唤,向32级前进}
    次日,小床发表视频%
    \href{https://www.bilibili.com/video/BV1zG4y197cG}{\cjkul{《这下不得不玩原神了...》}}\CJKecglue\footnote{BV1zG4y197cG}\CJKecglue。
    在登录账号阶段小床称他用了之前因为崩坏2注册的米哈游通行证,这带来了很微妙的暗示\pozhehao{}友善解读:小床是米元老;恶毒解读:小床为之后可能会发表的批评言论提前叠甲。
    
\end{multicols*}











\stdlastpage{2022-2023}

本期内容作者列表:\\
Neruthes /
Neruthes /
Neruthes /
Neruthes /
Neruthes /
Neruthes

\vskip 10pt

图片来源:\\
Page 3 Pic 1 ``Photo Title'' by Firstname Lastname /
Page 3 Pic 1 ``Photo Title'' by Firstname Lastname /
Page 3 Pic 1 ``Photo Title'' by Firstname Lastname /
Page 3 Pic 1 ``Photo Title'' by Firstname Lastname /
Page 3 Pic 1 ``Photo Title'' by Firstname Lastname /
Page 3 Pic 1 ``Photo Title'' by Firstname Lastname /
Page 3 Pic 1 ``Photo Title'' by Firstname Lastname
\end{document}
