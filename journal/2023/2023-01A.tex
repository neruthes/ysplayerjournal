\documentclass[11pt,a4paper]{report}
\usepackage[a4paper,textwidth=42em,tmargin=15mm,bmargin=28mm,footskip=15mm]{geometry}

\usepackage{calc}
\usepackage[dvipsnames]{xcolor}
\usepackage{amsmath,amssymb,xltxtra,fontspec,xunicode,tocloft,fancyhdr,indentfirst}
\usepackage{graphicx,eso-pic,datetime2,tabu,tcolorbox,multicol}
\graphicspath{ {./large-assets}{./large-assets/journal/2023} }
\usepackage{paralist,enumitem,varwidth}
\setdefaultleftmargin{5em}{2em}{1em}{1em}{1em}{1em}

\usepackage[hidelinks]{hyperref}
\hypersetup{
	pdfauthor={Neruthes 等各个作者},
	colorlinks=false,
	pdfpagemode=FullScreen
}

\usepackage[PunctStyle=plain,RubberPunctSkip=false,CJKglue=\hskip 0pt,CJKecglue=\hskip 2.5pt plus 20pt]{xeCJK}
\usepackage{xeCJKfntef}
% \newcommand{\CJKecglue}{\hskip 2.5pt plus 20pt}

\setmainfont{BaskervilleF}
\setromanfont{BaskervilleF}
\setsansfont{Mona-Sans}
\setmonofont{JetBrains Mono NL}
\setCJKmainfont{Noto Serif CJK SC}
\setCJKromanfont{Noto Serif CJK SC}
\setCJKsansfont{Noto Sans CJK SC}
\setCJKmonofont{Noto Sans CJK SC}
% =========================================


\linespread{1.2}
\setlength{\parindent}{2em}
\setlength{\parskip}{0pt}
\setlength{\columnsep}{2em}
% \setlength{\columnseprule}{0.4pt}

% Playing with tocloft
\renewcommand{\contentsname}{\LARGE\sffamily\bfseries 本期看点~~~~\Large\mdseries TBALE OF CONTENTS}
\renewcommand{\cftchapfont}{\sffamily\small}
\settowidth{\cftchapnumwidth}{\cftchapfont 2222}
\renewcommand{\cftchappagefont}{\sffamily\small}
\setlength{\cftbeforechapskip}{1pt}









\newcommand{\isDraft}[0]{
	\AddToShipoutPictureFG*{%
		\put(30mm,30mm){%
			\rotatebox{45}{\textcolor{red}{\fontsize{100pt}{100pt}\selectfont\rmfamily D~R~A~F~T}}
		}
	}
}
\newcommand{\maketitlepage}[7]{
	% argv: year, seq, textcolor, bgpicshift, issuetime, bgimg, briefing
	\hypersetup{pdftitle={尘世七国的记忆 - #5}}
\begin{titlepage}
	\newgeometry{textwidth=\textwidth+2em,tmargin=16mm,bmargin=16mm}
	\parindent=0pt
	\AddToShipoutPictureBG*{%
		\put(#4,-1mm){%
			% \includegraphics[height=\paperheight+2mm]{cover/cover-#1-#2}
			\includegraphics[height=\paperheight+2mm]{#6}
		}
	}
	\color{#3}
	\rule{\linewidth}{2.5pt}\vskip 15pt
	\begin{minipage}{\linewidth}
		\fontsize{45pt}{50pt}\selectfont
		\rmfamily\bfseries
		尘\hfill 世\hfill 七\hfill 国\hfill 的\hfill 记\hfill 忆
	\end{minipage}\par\vskip 10pt
	\begin{minipage}{\linewidth}
		\normalsize
		\sffamily\bfseries
		那些关于回旋镖、照妖镜、时光机的故事\hfill
		\LARGE#5
	\end{minipage}\par\vskip 10pt
	\rule{\linewidth}{2.5pt}\vskip 55pt

	\begin{minipage}{\linewidth}
		#7
	\end{minipage}

	\vfill
	\begin{minipage}{\linewidth}
		\footnotesize\sffamily
		主编 Neruthes\hfill 文章版权归作者\hfill CC BY-NC-ND 4.0
	\end{minipage}
	\rule{\linewidth}{2.5pt}
\end{titlepage}
\restoregeometry
\color{black}
\fancypagestyle{plain}{
    
	\renewcommand{\headrulewidth}{0pt}
	\renewcommand{\footrulewidth}{0.4pt}
	\lhead{}
	\chead{}
	\rhead{}
	\lfoot{\footnotesize \textbf{尘世七国的记忆}\hskip 2em#5}
	\cfoot{}
	\rfoot{\small\thepage}
}
\pagestyle{plain}\sffamily
}





\newcommand{\coverstory}[2]{
	% argv: author, title
	\begin{center}
		\sffamily\small\bfseries 封面故事~~/~~#1\\
		\rmfamily\Huge#2
	\end{center}
}
\newcommand{\dividearticles}[0]{
	\vskip 30pt plus 30pt minus 5pt
	\noindent\rule{\linewidth}{3.5pt}
	\vskip 25pt plus 25pt minus 5pt
}
\newcommand{\iarticle}[4]{
	% argv: author, title, tagline, noteline
	% \vskip 50pt
	\stepcounter{chapter}
	\setcounter{section}{0}
	\phantomsection\addcontentsline{toc}{chapter}{\numberline{\thechapter}#2}
	\noindent\parbox{\linewidth}{
		{\small\sffamily\bfseries#3}\par\vskip 7pt\nopagebreak
		{\LARGE\rmfamily\bfseries#2}\par\vskip 3pt\nopagebreak
		{\footnotesize\rmfamily\mdseries#4}\par\nopagebreak
		\rule{\linewidth}{0.4pt}\nopagebreak
	}\par\nopagebreak
	\vskip 15pt\nopagebreak
}
\newcommand{\isecnum}[0]{
	\stepcounter{section}
	\mbox{\hbox to 21.9pt{\arabic{section}. \hfill}}
}
\newcommand{\isection}[1]{
	% argv: sectionname
	\vskip 22pt plus 40pt minus 5pt
	\noindent\parbox{\linewidth}{\Large\rmfamily\bfseries#1}
	\vskip 15pt plus 30pt minus 5pt
}
% \newcommand{\notebox}[2]{
% 	\begin{tcolorbox}[colback=gray!5!white,colframe=gray,boxrule=0.4pt,width=\linewidth,left=5pt,right=5pt,top=5pt,bottom=5pt,sharp corners]
% 		\fontsize{10pt}{11pt}\selectfont
% 		#2
% 	\end{tcolorbox}\par
% }
\newcommand{\paraimg}[1]{%
        \vskip 0pt plus 30pt
        \noindent\includegraphics[width=\linewidth]{#1}
        \vskip 3pt plus 30pt
    }
\newcommand{\flisthref}[2]{%
    \noindent%
    \parbox{\linewidth}{
        \raggedright
		\fontsize{9pt}{12pt}\selectfont%
		\baselineskip=10pt
		\href{#1}{\rmfamily#2}\hfill\nopagebreak%
		\\
		\fontsize{8pt}{8.5pt}\selectfont%
		\href{#1}{\sffamily\selectfont\mbox{\textcolor{blue!35!gray}{[~#1~]}}}%
		\hfill
	}\vskip 5pt
}




\newcommand{\stdlastpage}[2]{
	% argv: yearRange, extraContent
	\clearpage\pagestyle{empty}
	\leavevmode\vfill
	\noindent\begin{minipage}{382.5pt}
		\fontsize{8.5pt}{10.5pt}\selectfont
		\parindent=0pt
		Copyright \copyright{} #1 Neruthes and other contributors.\\
		版权所有 \copyright{} #1 Neruthes 与其他贡献者。\\
		文章版权归各个作者所有。作者文章内容不代表本刊立场。\\
		本刊独立运作,与米哈游没有任何隶属、指导、赞助关系。\\
		此文档内包含《原神》游戏画面之截图、官方宣传图片,其版权属于上海米哈游影铁科技有限公司或其关联单位。
		本刊在版权法「合理使用」框架内谨慎使用此类素材。
		此文档内可能包含其他图片,具体版权情况以此文档内的备注为准。\\
		此文档以\CJKecglue\href{https://creativecommons.org/licenses/by-nc-nd/4.0/}{\CJKunderline{CC BY-NC-ND 4.0}}\CJKecglue{}许可证发布。\\
		任何人和单位可以在非商业使用、不修改的前提下复制、印刷、分发此文档(PDF)。\par
		\vskip 10pt

		本期刊在GitHub上的项目仓库:\href{https://github.com/neruthes/ysplayerjournal}{https://github.com/neruthes/ysplayerjournal}\\
		有关投稿方式等信息,请参阅项目仓库首页的README.md文件。
		\par
		\vskip 10pt

		#2
	\end{minipage}
}
\newcommand{\slashsep}[0]{\hskip 7pt/\hskip 7pt}








% ============================
% Miscellaneous utilities
\newcommand{\cjkul}[1]{\CJKunderanyline{0.5ex}{\rule{2pt}{0.4pt}}{#1}}
\newcommand{\pozhehao}[0]{{\fontspec{FandolSong}\CJKfontspec{FandolSong}——}}
% \newcommand{\pozhehao}[0]{\mbox{\raisebox{0.5ex}{\rule{2em}{0.4pt}}}}


\begin{document}
\isDraft
% \maketitlepage{2023}{01}{white}{-195mm}{2023 年 01 月~(上)}{cover-raw/3.3-akievent-splash}{
\maketitlepage{2023}{01}{white}{-1mm}{2023 年 01 月~(上)}{cover/3.3-akievent-splash}{
	\coverstory{Neruthes}{主编创刊寄语}

	\coverstorysmallleft{行秋传说任务第 1 幕与暴雪网易分手事件}\par
	\coverstorysmallleft{\setfontsize{23pt}3.3 剧情非常大的疑点:「国崩」去哪儿了}\par
	\coverstorysmallleft{Bilibili 用户「我牌哪去了」年底视频瓜}\par
	\coverstorysmallleft{\setfontsize{32pt}TGA 玩家之声奖事件}\par
	\coverstorysmallleft{三个小妖怪与面具建模相关争议}\par
}
\noindent\includegraphics[width=\linewidth]{./large-assets/decoration/banner-2400x300-001}\par
\tableofcontents\clearpage






\dividearticles
\iarticle{Neruthes}{主编创刊寄语}{\#编辑部公告}{作者~/~Neruthes \hfill 2023/01/01~~~All rights resered}

\isection{记录的意义}
2022年就要结束了,提瓦特的旅途大约已经过半,我们在蒙德、璃月、稻妻、须弥留下了弥足珍贵的回忆。
在游戏本体之外,节奏和4D剧情是游戏乐趣的重要组成部分,同样值得铭记。
在魔神任务第3章第5幕《虚空鼓动,劫火高扬》、魔神任务间章第3幕《倾落伽蓝》,米哈游向我们展现了岁月史书巨大威力。
将历史真相(包括争议中不同观点对撞的场面)记录下来,不仅有吃瓜的美味,更是遨游星海之人与摩拉克斯的契约。

\isection{个人的爱好}
今年,我更加深入地体验了LaTeX,这是比HTML+CSS更加优雅、更加强大的排版工具链。
虽然之前\href{https://neruthesgithubdistweb.vercel.app/qyxt/qyxt/qyxt-2022-02.pdf}{\CJKunderline{关于促进群友交流的尝试}}%
一直未能达到预期的效果,但那段经历让我对LaTeX工具链的熟悉程度提升了很多。

\isection{欢迎来投稿}

本刊接受如下投稿方式:

\begin{compactitem}
	\item \href{https://github.com/neruthes/ysplayerjournal}{\CJKunderline{在GitHub开issue}},将投稿内容写入issue正文。
	\item 在Telegram上联系\CJKecglue\href{https://t.me/neruthes}{\CJKunderline{@neruthes}},提供稿件内容(请不要发附件)。
	\item 以纯文本形式向\CJKunderline{i@neruthes.xyz}发送电子邮件,将投稿内容写入正文。
\end{compactitem}

向本刊投稿,即代表你接受以下条件:

\begin{compactitem}
	\item 你同意你对本刊作出的贡献将会以上文「版权声明」章节规定的形式向第三方授权。
	\item 本刊在接受你的投稿时可能对文本内容做出小幅修订(包括但不限于:错别字,拼写错误,成语误用)。
	\item 本刊主编有权指定下一任主编,由其继承主编在维护、编撰本刊范围内形成的一切权利(包括但不限于:指定继任者的权利,基于你过去的投稿内容推出新期刊、修订往期期刊、编撰合订本的权利)。
\end{compactitem}






\realpart{玩家社区}
\dividearticles
\iarticle{Neruthes}{行秋传说任务第1幕与暴雪网易分手事件}{\#时光机}{作者~/~Neruthes \hfill 2023/01/01首发本刊~~~此文以 CC BY-NC-ND 4.0 许可证发布}

\begin{multicols}{2}
	\paraimg{2301A/vidcover-zyh1}

	近日,Bilibili用户「自由魂儿儿儿」开播原神后,在某次直播中做到了行秋传说任务第1幕,感慨剧情走向十分接近最近暴雪网易分手事件。
	主播及观众纷纷表示逢魔大伟有时光机。

	\vskip 10pt
	\flisthref{https://www.bilibili.com/video/BV1gG411T7Cr}{逢魔大伟实锤了!网易暴雪分手就是他干的!}
\end{multicols}



\dividearticles
\iarticle{Neruthes}{Bilibili 用户「我牌哪去了」年底视频瓜}{\#切瓜}{作者~/~Neruthes \hfill 2023/01/01首发本刊~~~此文以 CC BY-NC-ND 4.0 许可证发布}

\noindent
我们将事件的主角简称为「小床」。这是他本人认可的官方昵称。

\begin{multicols*}{2}
	\isection{\isecnum 前情提要:开播原神之前}

	\paraimg{2301A/vidcover-bed1}

	小床在开播原神之前有另一个账号「我床哪去了」(所以由此得到「小床」之昵称)。老账号在开播原神前已经积累大约7.5万关注者。
	老账号在2022年9月3日发表视频\href{https://www.bilibili.com/video/BV12e4y1Y71p}{\cjkul{《这下不得不聊一下原神了…》}}。

	该视频内容比较「理中客」,站稳「局外人」立场,不评游戏只评节奏,看起来像是打算试探其既有粉丝是否支持向原转型。然后告一段落。

	小床建号「我牌哪去了」,并于2022年11月30日发表了第一个视频%
	\href{https://www.bilibili.com/video/BV1XG411M7Nh}{\cjkul{《原神【七圣召唤】不开天梯是好事吗?》}}。

	这个视频延续之前的风格,属于「局外人臆测」的立场,客观地或至少看似客观地探讨了天梯等机制在七圣召唤中的取舍问题。

	此视频内容可称无功无过,但与当时各个玩家社区的天梯话题之争议漩涡形成了共鸣,所以引发了一点争议。当时天梯、段位话题下主要有两派意见。

	一派认为,游戏内排位属于资格证,没有段位会让玩家社区难以基于发言人的段位评估其战术理解的参考价值,让社区的交流更容易被低段位玩家「污染」,
	并且不喜欢PvP的传统米氏PvE玩家可以选择不玩天梯排位只打快速匹配。
	另一派认为,米哈游不忘初心,始终记得原神是一个PvE游戏。

	\isection{\isecnum 第一阶段:为七圣召唤,向32级前进}
	次日,小床发表视频%
	\href{https://www.bilibili.com/video/BV1zG4y197cG}{\cjkul{《这下不得不玩原神了…》}}。
	在登录账号阶段小床称他用了之前因为崩坏2注册的米哈游通行证,这带来了很微妙的暗示\pozhehao{}友善解读:小床是米元老;恶毒解读:小床为之后可能会发表的批评言论提前叠甲。

	从P1到P9,小床的游玩视频得到了相当积极的评价,被许多网友称为「电子榨菜」,形容其「下饭」。
	小床以较好的记忆力(能记住之前剧情里的对话)、幽默风趣的吐槽(大号派蒙)、仔细阅读文本了解世界观设定的认真态度,为其争取了许多好评。

	12月上半,小床的新号「我牌哪去了」快速积累粉丝。截至2022年12月17日,已有12.2万。

	\isection{\isecnum 第二阶段:榨菜变味}

	大约在P10或P11开始,小床对魔神任务第1章第1幕《浮世浮生千岩间》至第1章第3幕《迫近的客星》表现出了烦躁。
	烦躁的表现具体包括:剪掉大世界游玩的许多场面,急于找刺杀岩王帝君的凶手,漏掉北国银行内钟离的对话。
	这引发了一些争议。

	一派认为:
	魔神任务第1章确实写得有大坑,叙事节奏错乱、铺垫不足,叙述性诡计没发挥好,立意很高但写崩了。
	另一派认为:
	花神诞祭诚不我欺,一周后就该车稻妻的主线剧情了。

	2022年12月17日,笔者整理史料时,发现NGA原版许多讨论小床的帖子遭到了锁隐。
	本文难以在可靠的史料上考证,许多信息总结自笔者的回忆。

	\isection{\isecnum 相关史料}

	\subsection*{NGA}

	\flisthref{https://ngabbs.com/read.php?tid=34712534}{[闲聊杂谈] 发生甚么事了,为什么又忽然开始车璃月剧情?花神诞祭又开始了?}
	\flisthref{https://ngabbs.com/read.php?tid=34705205}{[剧情讨论] 历史是个轮回,车了一圈剧情又车回璃月了}
	\flisthref{https://ngabbs.com/read.php?tid=34710527}{[闲聊杂谈] 关于电子榨菜变味这件事}
	\flisthref{https://ngabbs.com/read.php?tid=34704996}{​[闲聊杂谈] 榨菜水,谈一下他觉得剧情最大的不合理之处,其实是基于战力上的分析}

	\subsection*{Bilibili}

	\flisthref{https://www.bilibili.com/video/BV1XG411M7Nh}{原神【七圣召唤】不开天梯是好事吗?}
	\flisthref{https://www.bilibili.com/video/BV1zG4y197cG}{这下不得不玩原神了…}
	\flisthref{https://www.bilibili.com/video/BV11e4y1M7CU}{这下不得不玩原神了…【P5读书人的事怎么能叫偷呢?】}
	\flisthref{https://www.bilibili.com/video/BV1wM411r7B3}{这下不得不玩原神了…【P6你为什么这么瘦?】}
	\flisthref{https://www.bilibili.com/video/BV1FV4y1P7za}{这下不得不玩原神了…【P6.5-7也得给派蒙找个学上啊】}
	\flisthref{https://www.bilibili.com/video/BV11G411T7pW}{这下不得不玩原神了…【P8再见 蒙德】}
	\flisthref{https://www.bilibili.com/video/BV1iR4y1C7yi}{这下不得不玩原神了…【P9-9.5我只是一个路过的蒙面英雄】}
	\flisthref{https://www.bilibili.com/video/BV1pK4119733}{这下不得不玩原神了…【P10长官 我真是冤枉的】}
	\flisthref{https://www.bilibili.com/video/BV1QV4y1w7LX}{这下不得不玩原神了…【P11一口气全跑完】}
	\flisthref{https://www.bilibili.com/video/BV1544y1U7cV}{这下不得不玩原神了…【P12这就是当富哥们的感觉吗?】}
	\flisthref{https://www.bilibili.com/video/BV1aD4y187Xa}{这下不得不玩原神了…【P13没想到我还有唱歌的天赋】}
	\flisthref{https://www.bilibili.com/video/BV1Hg411E7Vq}{这下不得不玩原神了…【P14只有我受伤的世界】}
	\flisthref{https://www.bilibili.com/video/BV18K411r79t}{【小床】【原神璃月剧情复盘】}
	\flisthref{https://www.bilibili.com/video/BV1wW4y1T7wW}{群玉阁:哎 我又回来了【这下不得不玩原神了…】P15}
\end{multicols*}



% \clearpage
% \dividearticles
% \iarticle{咖喱芹菜}{关于电子榨菜变味这件事}{\#闲聊杂谈}{作者~/~咖喱芹菜 \hfill 2022/12/16首发于NGA~~~授权本刊收录}
% \begin{multicols}{2}
% 	关于最近比较火的叫做小床的up玩原神的开荒视频(电子榨菜),我也在看而且喜欢看。

% 	实话说吧,最开始看他是因为他前期吐槽原神的那些点,实际上几乎都是会被后期原神的进步打脸的。所以我想看他继续玩下去然后真香。

% 	现在有点不好看,倒不是因为他不够「真香」(毕竟还在铺垫阶段),而是他的一些粉丝离谱的「护主」行为让我觉得看他反应变得很无趣了(我在这段文字的最后解释了一下这种无趣的感觉是什么样的)。

% 	他的璃月篇的下饭点在于:为了凹「高智商分析型up」的人设,他在很努力地分析,结果因为原神主线里那些为了让漫不经心的玩家不太无聊而经常出现的无厘头搞笑片段和「恶意」玩梗桥段,以及他自己心里有一套自己主观的「游戏剧情该怎么做」的刻板印象(中性词),导致他的猜测跑偏了。

% 	这本身是很有节目效果的。最后他一波「被演了」的懊恼,也很有喜剧感。反正我在不看评论区之前,看他吐槽或者被无厘头桥段心累到整得瘫在椅子上,是看得挺开心的。

% 	这种时候本来大家就善意调侃一下他,然后接着看他玩就完事了。毕竟他只要是个正常人,后面玩多了,就会想明白他现在吐槽的这些点其实都是无关紧要的。原神本身的质量是非常过硬非常能打的,不管是「担心他弃坑所以疯狂给他攻略」还是「担心他误会原神所以疯狂解释」其实都是毫无必要,甚至有点蠢的行为。

% 	既不能让榨菜更香,更让他产生了「你们这是把我当不会玩游戏的睿智一样哄」的烦躁感和叛逆感。

% 	看的不就是他自己发现原神其实很牛的这个过程吗?粉丝们手把手喂给他只是满足自己养成的乐趣吧,去找那些适合养成的up不就好了吗?自己唠唠叨叨又生怕说错话让他不开心的样子,不觉得像是「为你好」然后给你一堆人生大道理的长辈亲戚(老爷子老妈子)一样烦吗?

% 	至于剧情主线,原神本来一开始就没有主打「硬核剧情向」。主线为了让「疯狂跳过党」也能大概明白个123,以及要让各个角色出场,实际上是整个原神里逻辑最幼稚、叙事最墨迹的剧情。这已经是公认的事情了吧?中间搞几个记忆点,也不过就是给一直点点点的人找点乐子,让他们不至于直接睡过去。

% 	这是免费游戏为了适应广大的玩家的妥协。剧情向付费单机游戏玩家确实不会有这个问题,毕竟花了钱的人多少会认真看。但原神他不靠门票赚钱,那就什么人都要尝试拉拢一下。爱挖掘剧情的人是有很多能挖掘的,原神里给的暗示简直不要太多了。

% 	这种事情跟小床费再多口舌也没啥用,他只要不是傻子,自己多玩一会就自己能明白。(不用担心他不玩,他现在这么火怎么可能不玩?跟钱过不去吗?)

% 	但小床不知道何时形成的粉丝开始无脑「床宝说的对,璃月剧情就是很糟,稻妻剧情更糟,balabala」。生怕惹到小床不高兴。啊这?大家怎么都成鉴赏家了,原神这主线剧情在开放世界游戏里怎么也算得上是一般水平、也不是很逆天吧。我怎么不记得有哪个开放世界是靠主线剧情出彩的了?就连剧情取胜的巫师三主要也是支线备受好评,主线也不就是平庸的「找A解决一件事,找B解决一件事,找C发现另一件事,然后D又跑出来让主角去找E」之类的跑腿任务?又开始犯「只要原神哪一点不是最高水平,那它就是辣鸡」的这种陈年老节奏了吗?

% 	最重要的是,璃月剧情怎么样,和他因为自己猜错了而在那里低气压到底有什么关系?对后者的行为,尽情笑就完了,怎么还在那里反思的反思,安慰的安慰?

% 	很搞笑的事情被严肃地拿来讨论「谁对谁错」,甚至还上升到游戏本身,这真的太无趣了。
% 	这种感觉就好像你正嘲笑你一个同事因为睡过头而被老板抓包,结果他女朋友义正严辞地谴责公司不考虑员工通勤状态把开会时间定那么早,又说公司不该早上开会是形式主义,最后又很正经地说其他公司福利更好。

% 	气氛是不是一下子就冷了?重点真的是公司是怎么样吗?谁在乎啊?最开始不只是想调侃他迟到被抓包的糗样而已吗?

% 	言尽于此。

% 	\vskip 20pt plus 20pt
% 	\flisthref{https://ngabbs.com/read.php?tid=34710527}{[闲聊杂谈] 关于电子榨菜变味这件事}
% \end{multicols}


% \clearpage
\dividearticles

\iarticle{Neruthes}{TGA玩家之声奖事件}{\#回旋镖~~~~\#照妖镜}{作者~/~Neruthes \hfill 2023/01/01首发本刊~~~此文以 CC BY-NC-ND 4.0 许可证发布}
\begin{multicols*}{2}
	\isection{\isecnum 前情提要}
	TGA全名The Game Awards,号称游戏界奥斯卡。虽然近年来争议不断,但仍然是有较高影响力的奖项评选活动。
	TGA 2022设置了「玩家之声」奖项,提名了5款游戏:

	原神、战神5诸神黄昏(God of War: Ragnar\"ok)、艾尔登法环(Elden Ring)、索尼克边境地带(Sonic Frontiers)、迷失(Stray)。

	\isection{\isecnum 半场香槟}
	玩家之声奖开始投票后一段时间,索尼克边境地带处于领先地位。于是,部分「极端索尼克粉丝」在Twitter上发表了一些「令人感到不安」的图片。
	其中一张获得了较多注意的图片内容是索尼克持械殴打荧。

	\isection{\isecnum 矛盾初现}
	在原神玩家社区,新的风暴开始酝酿。
	开始有一些声音动员原神玩家参加投票\pozhehao{}得奖事小,争这口气很有意义。

	\isection{\isecnum 刷票风波}
	有人注意到,索尼克边境地带的票数出现了不正常的增长。
	并且,Twitter上有人声称他(及其同僚)在研究TGA是否存在刷票漏洞,索尼克边境地带突然出现的1000票可能是实验的副作用。

	\isection{\isecnum 尘埃落定}
	主办方将玩家之声奖颁发给了原神。难绷的是,主持人颁奖时特意强调了「在移除所有机器人投票后」。

	而后,英文社区出现了弔图,其内容大致是:
	Never ask a man his salary, a woman her age, a Sonic Frontier fan
	what happened on Thursday December 8th, 2022.

	之后,还有另一派「极端索尼克粉丝」绘制「荧与索尼克拥抱亲吻」主题的画作,引发了更多争议。
	原神玩家社区有一种声音是,狗皮膏药一样换不同姿势贴原,真烦。

	\isection{\isecnum 勉强相关}

	曾担任索尼克系列游戏制作人的「索尼克之父」中裕司,近日在东京因为涉嫌股票内幕交易被逮捕。
	NGA瓜版的猹们纷纷表示米氏照妖镜显灵了。

	\isection{\isecnum 相关史料}

	\flisthref{https://ngabbs.com/read.php?tid=34569740}{[外服消息] 外网部分玩家P图拉踩原神的手段真是太低劣了}
	\flisthref{https://ngabbs.com/read.php?tid=34560157}{[原神] 索尼克粉丝称原神贿赂粉丝以获得TGA投票}
	\flisthref{https://ngabbs.com/read.php?tid=34569524}{[新闻相关] TGA玩家之声决赛,原神获奖!}
	\flisthref{https://bbs.nga.cn/read.php?tid=34654423}{[闲聊杂谈] 索尼克粉开始幻想和原神联动}
	\flisthref{https://www.bilibili.com/video/BV1x14y1J7Dk}{[夜泽君pro] 污蔑原神刷票?P鬼图侮辱荧妹?索尼克粉成功激怒了原神玩家!TGA投票原神以压倒性优势反杀索尼克!【快讯】}
\end{multicols*}

\clearpage
\dividearticles

\iarticle{Neruthes}{三个小妖怪与面具建模相关争议}{\#争议}{作者~/~Neruthes \hfill 2023/01/01首发本刊~~~此文以 CC BY-NC-ND 4.0 许可证发布}
\begin{multicols}{2}
	3.3版本活动试胆大会的第2天,有三个小妖怪登场。三个小妖怪的建模引发了一些争议。部分人士提出了这样的批评意见:

	建模过于节约,作为玩家难以(像剧中人物一样)分辨妖怪与戴面具的小孩,玩家与剧中人物对视觉信息的认知差异过大,剧情体验不好。
	若与过去夜兰建模争议放在一起考虑,合订本很难看。

	针锋相对地,讨论中出现了另一种意见:

	首先,过去的文艺作品中(例如夏目友人帐),很多妖怪都是人形戴面具,非人化程度并不高。
	其次,非人化程度再高一些(但不到特瓦林、安德留斯、若陀龙王的程度),容易恐怖谷,会很掉SAN,不符合本游戏的12+评级。

	\isection{相关史料}

	\flisthref{https://ngabbs.com/read.php?tid=34724500}{[闲聊杂谈][剧透预警] 我觉得戴面具已经是美术的妥协了}
\end{multicols}

% \clearpage
\dividearticles
\iarticle{是一只斑鸠}{3.3剧情非常大的疑点:「国崩」去哪儿了}{\#原学研究}{作者~/~是一只斑鸠 \hfill 2022/12/09首发于NGA~~~授权本刊收录}
\begin{multicols*}{2}
	我们翻流浪者的角色故事,可以看到他的人生有以下几个阶段:

	诞生:并没有得到赐名

    \paraimg{2301A/kunikuzushi-1}

	捡回踏鞴砂:倾奇者(依旧没有名字)

    \paraimg{2301A/kunikuzushi-2}

	小孩死后:恢复到无名无姓的状态,开始流浪

    \paraimg{2301A/kunikuzushi-3}

	带回愚人众:人偶(没给自己起名字)

    \paraimg{2301A/kunikuzushi-4}

	锄完深渊:散兵

    \paraimg{2301A/kunikuzushi-5}

	那么问题来了:

    \paraimg{2301A/kunikuzushi-6}

	在流浪的最后,他给自己起名「国崩」,这和角色故事相冲突。
	更重要的,那个自己都不记得的曾经用过的名字全程没提。
	「国崩」这个名字,曾经用过的名字。无论是在语音,角色故事,还是3.3间章,都神隐了。
	\paraimg{2301A/kunikuzushi-7}

	只有3.2他自己说了一句「名为国崩的一切,都是如此渺小,如此丑陋」。
	而他自己是怎么评价自己的人生阶段的呢?
	「人偶是被遗弃的懦夫,倾奇者是遭人包庇的无为者,斯卡拉姆齐是密谋者」。

	国崩呢?国崩去哪儿了?

	兜兜转转,回到容彩祭,相信大家都还记得那些问题:
	国崩到底是谁;
	听神里绫人说,前来寻找秘密的是愚人众的密探;
	究竟是谁想要隐藏他的过去;
	他的秘密,也会对你的命运造成影响吗。

	容彩祭纸条只是内容改变,但依旧没有改变愚人众前来偷纸条被绫人察觉这件事。否则容彩祭事件不会发生。
	既然是百目家的怨种干的,愚人众来偷什么信呢?

	以及3.2的:
	「愚人众要带走海芭夏,是想让给散兵保密吗?」

	我怀疑:
	流浪者现在的记忆并不完整,他现在的记忆是教院给他抽出来,然后草神帮他塞回去,如果说,那段被抽出来的记忆本身就不完整呢?

	或者说阴谋论一点:
	「在征服神之目光以后,他将迈出新的一步」愚人众料到他会取得神之眼。那草神读到的博士的记忆,会不会也是故意的?
	抽散兵记忆也是故意的,而被抽的那段记忆少了「关于国崩的具体故事」「关于被他自己忘掉的名字」的,这一段正好是愚人众「想要掩盖的秘密」。

	而绫人在容彩祭的表现也不会是无用功,他的调查可以牵扯到剧情大药。
	那个被水泼了才会显形的屏风被拿回社奉行了,世界树的覆盖会不会和这个屏风有某种程度的相似呢?
	世界树无法覆盖掉小草的童话故事,如果说,在现在这个世界线,拿水泼这个屏风,会发生什么呢?

	以上,自己的一些小推测。

	\vskip 10pt plus 10pt
	\textbf{事后编辑}:

	其实支撑我推出这一点的是流浪者对雷电五传的态度。
	正常逻辑下,得知真相的流浪者最该愧疚的最想要弥补的应该是死掉的雷电五传的那些人。

	但是我仔细翻了一遍主线和角色故事,他对雷电五传给我的一种感觉就是「不关心,来清算我也欢迎」。

	他删自己的目的是拯救当年死在踏鞴砂的人。是的,只有丹羽桂木他们那一批刀匠和居民。
	他希望通过改变自身来影响博士的行动,给他踏鞴砂当年的朋友们留出点活的可能,让他们过上另一种美好的人生。

	至于雷电五传……一句都没提,他删自己的时候压根都没想到雷电五传……
	而且,这么大的事情连住进他的角色语音和角色故事的资格都没有,在间章PV他的记忆闪回那里,连一个不知名路人都有镜头,但是雷电五传却没有。
	很奇怪不是吗?这不应该是他人生阶段中相当重要的一个部分吗?
	而与雷电五传强相关的名字是哪个呢?\textbf{国崩}。

	\vskip 30pt plus 30pt
	\flisthref{https://ngabbs.com/read.php?tid=34620957}{[剧情讨论] 3.3剧情非常大的疑点:「国崩」去哪儿了}
\end{multicols*}


\realpart{论坛拾枝}
\realpartheader{论坛拾枝}
\begin{multicols}{2}
	\mininews{意大利情景喜剧除了11名执行官以外的一个名字代号}{
		\small 作者:密特菈斯\par
		\href{https://ngabbs.com/read.php?tid=34756975}{https://ngabbs.com/read.php?tid=34756975}
	}
	\mininews{提瓦特全历史梳理猜想:「转世说」与「轮回说」}{
		\small 作者:胆大龟\par
		\href{https://ngabbs.com/read.php?tid=34680515}{https://ngabbs.com/read.php?tid=34680515}
	}
	\mininews{隐藏在契约与岩之下\pozhehao{}对摩拉克斯真正权柄和身份的分析与猜想}{
		\small 作者:自传之轮\par
		\href{https://ngabbs.com/read.php?tid=33583183}{https://ngabbs.com/read.php?tid=33583183}
	}
	\mininews{散兵剧情以及降临者推测旅行者经历}{
		\small 作者:xv2bing\par
		\href{https://ngabbs.com/read.php?tid=34655554}{https://ngabbs.com/read.php?tid=34655554}
	}
	\mininews{优秀的创作者应该懂得避嫌1}{
		\paraimg{2301A/mininews-acjade1}
		\paraimg{2301A/mininews-acjade2}

		\flisthref{https://ngabbs.com/read.php?tid=34757147}{[腾讯] 刺客信条手游UI与某手游相似}
	}
	\mininews{优秀的创作者应该懂得避嫌2}{
		\paraimg{2301A/gua-hlr-1}

		\flisthref{https://ngabbs.com/read.php?tid=34731173}{[新瓜] 绘旅人新卡名撞车明日方舟主线标题和原神世界任务}
	}
	\mininews{QQ群搜索不到「七圣召唤」的结果}{
        \small 作者:70年式悠久机关\par
        \href{https://ngabbs.com/read.php?tid=34746818}{https://ngabbs.com/read.php?tid=34746818}
	}
\end{multicols}









\stdlastpage{2022-2023}{
	本期内容作者列表:\\
	Neruthes\slashsep
    是一只斑鸠

	\vskip 10pt

	图片来源:\\
	Page 3 - Figure 1 ``视频封面'' by 自由魂儿儿儿 (合理使用)\slashsep
	Page 3 - Figure 2 ``视频封面'' by 我牌哪去了 (合理使用)\slashsep
	Page XX - Figure 1/2 ``刺客信条游戏截图1'' by Ubisoft (合理使用)\slashsep
	Page XX - Figure 3 ``新浪微博截图'' by 时空中的绘旅人官方新浪微博账号 (合理使用)
}
\end{document}




